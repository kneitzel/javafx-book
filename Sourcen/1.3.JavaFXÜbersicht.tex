\section{JavaFX Übersicht}
Dieses Kapitel gibt eine Übersicht über diverse Bestandteile von JavaFX.

\subsection{Aufbau Applikation}

\subsubsection{Application Class}
In JavaFX Applikation wird in der Regel eine Klasse erstellt, die von Application erbt.

In der main Methode wird dabei nur launch der Application Klasse aufgerufen. Dies initialisiert
die eigentliche Applikation und startet den JavaFX Application Thread, der für die Event
Abarbeitung verantwortlich ist.

Die Application Klasse hat drei wichtige Methoden, die überschrieben werden können.
\paragraph{init}
Die init Methode wird als erstes aufgerufen und kan für spezielle Initialisierungen
verwendet werden, ehe das Hauptfenster erstellt wird.

\paragraph{start}
Die Start Methode ist abstrakt und muss überschrieben werden. Diese Methode ist der
Haupteinstiegspunkt und bekommt den Hauptbereich übergeben, den man mit seinem Inhalt
füllen kann. Dies ist in der Regel das Hauptfenster der Applikation, aber je nach
verwendeter Plattform kann dies auch ein Bereich z.B. im Browser sein oder die Hauptanzeige
einer mobilen Applikation.

\paragraph{stop}
Die Stop Methode wird aufgerufen, wenn die Anwendung beendet wird, z.B. über einen
Platform.exit(int) Aufruf.

\begin{tabular}[h]{|p{2cm}|p{9cm}|}
\hline
\textbf{Wichtig} & Wenn die Applikation über System.exit(int) beendet wird, wird diese
Methode nicht aufgerufen! \\
\hline
\end{tabular}

\paragraph{Beispiel für eine Application Klasse}
 
\begin{lstlisting}
package helloworld;

import javafx.application.Application;
import javafx.scene.Scene;
import javafx.scene.layout.StackPane;
import javafx.stage.Stage;

public class HelloWorld extends Application {
    public static void main(String[] args) {
        launch(args);
    }

    @Override
    public void start(Stage primaryStage) {
        primaryStage.setTitle("Hello World!");
        StackPane root = new StackPane();
        primaryStage.setScene(new Scene(root, 300, 250));
        primaryStage.show();
    }
}
\end{lstlisting}

\subsubsection{FXML Loader}
Der FXML Loader lädt eine fxml Datei und generiert die Elemente, die darin beschrieben sind.


\subsection{Container}
Übersicht über die diversen Container von JavaFX.

\subsubsection{AnchorPane}

\subsubsection{BorderPane}

\subsubsection{Arcordion / TitledPane}

\subsubsection{ButtonBar / ToolBar / HBox / VBox}

\subsubsection{GridPane}

\subsubsection{FlowPane / TilePane}

\subsubsection{TabPane / Tab}

\subsubsection{SplitPane}

\subsubsection{ScrollPane}

\subsubsection{SplitPane}

\subsubsection{StackPane}

