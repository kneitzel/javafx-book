\section{JavaFX Übersicht}
Dieses Kapitel gibt eine Übersicht über diverse Bestandteile von JavaFX.

\subsection{Aufbau Applikation}

\subsubsection{Application Class}
In JavaFX Applikation wird in der Regel eine Klasse erstellt, die von Application erbt.

In der main Methode wird dabei nur launch der Application Klasse aufgerufen. Dies initialisiert
die eigentliche Applikation und startet den JavaFX Application Thread, der für die Event
Abarbeitung verantwortlich ist.

Die Application Klasse hat drei wichtige Methoden, die überschrieben werden können.
\paragraph{init}
Die init Methode wird als erstes aufgerufen und kan für spezielle Initialisierungen
verwendet werden, ehe das Hauptfenster erstellt wird.

\paragraph{start}
Die Start Methode ist abstrakt und muss überschrieben werden. Diese Methode ist der
Haupteinstiegspunkt und bekommt den Hauptbereich übergeben, den man mit seinem Inhalt
füllen kann. Dies ist in der Regel das Hauptfenster der Applikation, aber je nach
verwendeter Plattform kann dies auch ein Bereich z.B. im Browser sein oder die Hauptanzeige
einer mobilen Applikation.

\paragraph{stop}
Die Stop Methode wird aufgerufen, wenn die Anwendung beendet wird, z.B. über einen
Platform.exit(int) Aufruf.

\begin{tabular}[h]{|p{2cm}|p{9cm}|}
\hline
\textbf{Wichtig} & Wenn die Applikation über System.exit(int) beendet wird, wird diese
Methode nicht aufgerufen! \\
\hline
\end{tabular}

\paragraph{Beispiel für eine Application Klasse}
 
\begin{lstlisting}
package helloworld;

import javafx.application.Application;
import javafx.scene.Scene;
import javafx.scene.layout.StackPane;
import javafx.stage.Stage;

public class HelloWorldApp extends Application {
    public static void main(String[] args) {
        launch(args);
    }

    @Override
    public void start(Stage primaryStage) {
        primaryStage.setTitle("Hello World!");
        StackPane root = new StackPane();
        primaryStage.setScene(new Scene(root, 300, 250));
        primaryStage.show();
    }
}
\end{lstlisting}

An diesem Beispuel erkennen wir die wichtigen Punkte:
\begin{itemize}

\item Die Klasse \bf{HelloWorldApp} erbt von Application.

\item Die Methode \bf{main} ruft launch auf und übergibt dabei die Parameter.

\item Die Methode \bf{start} wird überschrieben und in der Methode wird beim Hauptfenster (primaryStage) werden Titel
und Inhalt gesetzt und das Fenster sichtbar gemacht.
\end{itemize}

\subsubsection{FXML Loader}
Der FXML Loader lädt eine fxml Datei und generiert Instanzen für die Elemente, die darin beschrieben sind.

Dabei wird ggf. auch eine Instanz des controllers erzeugt und in dieser Klasse im Nachganz Instanzvariablen befüllt,
die mit @FXML\footnote{@FXML ist eine Java Annotation, welche auf Felder und Methoden angewendet werden kann. Die so
markierten Felder werden dann vom FXMLLoader berücksichtigt bei der Initialisierung.} markiert oder public\footnote{Felder
sollten nie public sein sondern immer private oder maximal protected. Wenn ein zugriff von außen benötigt wird, dann 
sollten dazu Getter und/oder Setter verwendet werden.} sind.

\subsection{Container}
Im Folgenden findet sich eine Übersicht über die diversen Container von JavaFX. Dies dient nur der groben Orientierung
und kann ein Experimentieren nicht ersetzen. Ich kann nur empfehlen, mit allen Containern etwas 'zu spielen' um so ein
erstes Verständnis zu dem Control zu bekommen.

\subsubsection{AnchorPane}


\subsubsection{BorderPane}

\subsubsection{Arcordion / TitledPane}

\subsubsection{ButtonBar / ToolBar / HBox / VBox}

\subsubsection{GridPane}

\subsubsection{FlowPane / TilePane}

\subsubsection{TabPane / Tab}

\subsubsection{SplitPane}

\subsubsection{ScrollPane}

\subsubsection{SplitPane}

\subsubsection{StackPane}

