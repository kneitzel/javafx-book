\section{Vorwort}

JavaFX ist ein Framework zur Erstellung von plattformübergreifenden, grafischen Oberflächen.

Bis ca. 2014 waren AWT und Swing die führenden Frameworks zur Erstellung von graphischen Java Applikationen. Die JavaFX Entwicklung 
startete ca. 2007 und wurde Ende 2008 in einer ersten Version von Sun freigegeben.

\begin{wrapfigure}{l}{3cm}
    \includegraphics[width=3cm]{konrad}
\end{wrapfigure}

Ich beschäftige mich seid über 30 Jahren mit dem Thema Software Entwicklung und habe in der Zeit vor allem in C / C++, C\# und Java entwickelt.
Java auf dem Desktop war für mich zwar beruflich nie ein Thema, aber ich habe mich mit der Thematik intensiv auseinander gesetzt, da es bei
JavaFX dem Entwickler teilweise schwer gemacht wird, seine Applikation sauber zu strukturieren.

Des Weiteren gab und gibt es im Java Forum\footnote{Java Forum: https://www.java-forum.org} immer wieder Startprobleme bei der Einbindung von javaFX im Entwicklungsprozess.

Daher werde ich in dieser Dokumentation die folgenden Bereiche abdecken:

Schnelle Übersicht über die von mir empfohlene Nutzung von JavaFX mit Maven oder Gradle sowie mit IntelliJ.

Übersicht über die Panes von JavaFX zur Gliederung einer Oberfläche.

Übersicht über MVC und den damit verbundenen Problemen.

Übersicht zu MVVM mit mvvmFX.

\subsection{Schreibweisen}
\subsubsection{Hinweise und Empfehlungen}
Besondere Hinweise, Warnungen oder Empfehlungen finden sich wie folgt in diesem Buch:

\begin{tabular}[h]{|p{2cm}|p{9cm}|}
\hline
\textbf{Empfehlung} & Dies ist eine Empfehlung des Autors, die beachtet werden sollte! \\
\hline
\end{tabular}

\subsection{Code-Beispiele}
Code wird im Buch immer wie folgt dargestellt:

\textbf{javafx-buch/kapitel/some/path/HelloWorld.java}
\begin{lstlisting}
package helloworld;

import javafx.application.Application;
import javafx.fxml.FXMLLoader;
import javafx.scene.Parent;
import javafx.scene.Scene;
import javafx.stage.Stage;

import java.io.IOException;

public class HelloWorld extends Application {

    public static void main(final String[] args) {
        launch(args);
    }

    @Override
    public void start(final Stage primaryStage) throws IOException {
        Parent root = FXMLLoader.load(getClass().getResource("HelloWorld.fxml"));
        Scene scene = new Scene(root);
        primaryStage.setScene(scene);
        primaryStage.show();
    }
}
\end{lstlisting}


\subsection{Danksagung}
Ich danke allen, die mir bei diesem Werk geholfen haben ...

\subsection{Weitere Ressourcen}

\subsubsection{Ressourcen des Buches}
Die Ressourcen des Buches finden sich auf Github unter https://github.com/kneitzel/javafx-book.

\subsubsection{Blog}
Auf https://blog.kneitzel.de/ findet sich der Blog von mir. Dort finden sich einige Artikel rund um die Software Entwicklung.

\subsection{YouTube}
Auf YouTube finden sich von mir diverse Videos, in denen ich über Software Entwicklungsthemen rede.

