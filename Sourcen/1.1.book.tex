\section{Über dieses Buch}

\subsection{Schreibweisen}
\subsubsection{Hinweise und Empfehlungen}
Besondere Hinweise, Warnungen oder Empfehlungen finden sich wie folgt in diesem Buch:

\begin{tabular}[h]{|p{2cm}|p{9cm}|}
\hline
\textbf{Empfehlung} & Dies ist eine Empfehlung des Autors, die beachtet werden sollte! \\
\hline
\end{tabular}

\subsection{Code-Beispiele}
Code wird im Buch immer wie folgt dargestellt:

\textbf{javafx-buch/kapitel/some/path/HelloWorld.java}
\begin{lstlisting}
package helloworld;

import javafx.application.Application;
import javafx.fxml.FXMLLoader;
import javafx.scene.Parent;
import javafx.scene.Scene;
import javafx.stage.Stage;

import java.io.IOException;

public class HelloWorld extends Application {

    public static void main(final String[] args) {
        launch(args);
    }

    @Override
    public void start(final Stage primaryStage) throws IOException {
        Parent root = FXMLLoader.load(getClass().getResource("HelloWorld.fxml"));
        Scene scene = new Scene(root);
        primaryStage.setScene(scene);
        primaryStage.show();
    }
}
\end{lstlisting}

\subsection{Autor}
Paar Informationen über mich ... Muss ich mir noch überlegen ...

\subsection{Danksagung}
Ich danke allen, die mir bei diesem Werk geholfen haben ...

\subsection{Weitere Ressourcen}

\subsubsection{Ressourcen des Buches}
Die Ressourcen des Buches finden sich auf Github unter https://github.com/kneitzel/javafx-book.

\subsubsection{Blog}
Auf https://blog.kneitzel.de/ findet sich der Blog von mir. Dort finden sich einige Artikel rund um die Software Entwicklung.

\subsection{YouTube}
Auf YouTube finden sich von mir diverse Videos, in denen ich über Software Entwicklungsthemen rede.

